%\documentclass[lang=en,aspectratio=43]{mupresentation}
\documentclass[lang=eu,biz=pls,aspectratio=169,handout]{mupresentation}

\title{Devcontainer-ak\\Informatika Graduan}
\subtitle{INFOR Ekimenak}
\institute{Mondragon Unibertsitatea}
\date{}
\version{v0.5.0}

\begin{document}

\mucover

\begin{frame}
  \frametitle{Zer da Container bat?}
  \begin{itemize}
    \item Container bat Linux kernel-eko prozesu \textbf{isolatu} bat da.
    \item Bere \textbf{fitxategi-sistema} propioa du.
    \item Fitxategi-sistema hau \textbf{erreproduzitu} genezake.
  \end{itemize}
\end{frame}

\begin{frame}
  \frametitle{Zer da Devcontainer bat?}
  \begin{itemize}
    \item Container-en teknologia erabiltzen dute \textbf{garapen inguruneak} sortzeko.
    \item Visual Studio Code-ren zerbitzari zatia Container barruan exekutatzen da.
    \item Erabiltzaileak beraien VSCode-rekin konekta dezakete bezero gisa. Horrela, \textbf{garapena Container-ea} egon balira bezala egin dezakete.
  \end{itemize}
\end{frame}

\begin{frame}
  \frametitle{Garrantzia eremu akademikoan}
  \begin{itemize}
    \item Ikasleek (eta irakasleek) garapen-ingurune ezberdinak erabiltzen dituzte graduan zehar.
    \item Devcontainer-ek ingurune hauek \textbf{bereizten} laguntzen dute.
    \item \textbf{Erreplikagarriak} dira. Ikasleek eta irakasleek ingurune berdina izan dezaten laguntzen dute.
  \end{itemize}
\end{frame}

\begin{frame}
  \frametitle{Zer egin dugu?}
  \begin{itemize}
    \item Ingurune hauek sortu eta banatzeko \textbf{git errepositorio} bat sortu dugu: \url{https://gitlab.com/mu-bd-ce/devcontainers}
    \item Devcontainer-ei buruzko \textbf{dokumentazioa} duen webgune bat garatu dugu: \url{https://mu-bd-ce.gitlab.io/devcontainers}
      \begin{itemize}
        \item Bertan nola erabili erakusteko \textbf{adibideak} daude.
      \end{itemize}
    \item Bigarren mailako web eta estatistika ikasgaietan \textbf{lehenengo saiakera} egin dugu.
  \end{itemize}
\end{frame}

\begin{frame}
  \frametitle{Gure Proposamena}
  \begin{itemize}
    \item Hurrengo kurtsoan, hurbildu nahi duten gaiek ekimen honetan \textbf{parte hartu} dezakete.
    \item Biltegiak \textbf{eztabaida-puntu} gisa funtzionatuko du (adibidez, graduko Java bertsioa).
    \item Hori lehenengo hurbilketa da; devcontainer-en alderdi guztiak eztabaidatzeko \textbf{irekita} daude.
  \end{itemize}
\end{frame}

\begin{frame}
  \frametitle{Laburbilduz}
  \begin{itemize}
    \item Devcontainer-ek garapen-inguruneak \textbf{estandarizatzeko} modu indartsua eskaintzen dute.
    \item \textbf{Kolaborazioa} eta \textbf{koherentzia} hobetzen dute akademikoen testuinguruan.
    \item Irakasleak ekimen honetan parte hartzera \textbf{gonbidatzen} ditugu hurrengo ikasturtean.
  \end{itemize}
\end{frame}

\muback{Eskerrik asko\\Muchas Gracias\\Thank you}{Software \& System Engineering\\
  Electronics and Computer Science department\\
  Mondragon Unibertsitatea - Faculty of Engineering\\
\href{www.mondragon.edu}{www.mondragon.edu}}

\end{document}