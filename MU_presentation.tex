%\documentclass[lang=en,aspectratio=43]{mupresentation}
\documentclass[lang=eu,biz=pls,aspectratio=169,handout]{mupresentation}

\title{Devcontainer-ak\\Informatika Graduan}
\subtitle{INFOR Ekimenak}
\institute{Mondragon Unibertsitatea}
\date{}
\version{v0.5.0}

\begin{document}

\mucover

\begin{frame}
  \frametitle{Zer da Devcontainer bat?}
  \begin{itemize}
    \item Software \textbf{garapena} egiteko behar den ingurunea barnean dakarren \textit{Docker} \textbf{Container} bat da.
    \item Visual Studio Code-ren zerbitzari zatia Container \textbf{barruan} exekutatzen da.
    \item Garatzailea bere sistema eragilean duen Visual Studio Code-rekin konektatu daiteke bezero gisa. Horrela, \textbf{garapena Container-ean} egon balitz bezala egiten da.
    \item Hasieran Microsoftek bultzatu zuen baina orain espezifikazio \textbf{ireki} bat da:
      \url{https://containers.dev/}
  \end{itemize}
\end{frame}

\begin{frame}
  \frametitle{Garrantzia graduan}
  \begin{itemize}
    \item Ikasleek eta irakasleek \textbf{hainbat} garapen-ingurune erabili behar dituzte graduan zehar. Ikasgai ezberdinak, transferentzia proiektuak, ikerketa, ...
    \item Devcontainer-ek ingurune hauek elkarren artean \textbf{bereizten} laguntzen dute.
    \item Devcontainer-ek ez dute sistema eragilea aldatzen.
    \item Bertsio ezberdinen arteko \textbf{talkak ekiditen} laguntzen dute.
    \item \textbf{Erreplikagarriak} dira. Ikasleek eta irakasleek ingurune berdina izan dezaten laguntzen dute.
  \end{itemize}
\end{frame}

\begin{frame}
  \frametitle{Zer egin dugu?}
  \begin{itemize}
    \item Devcontainer-ei buruzko \textbf{git biltegi} bat sortu dugu:
      \url{https://gitlab.com/mu-bd-ce/devcontainers}.

      Bertan aurkituko ditugu:
      \begin{itemize}
        \item Programazio lengoaia ezberdinendako \textbf{aurrez-sorturiko Devcontainer-ak}. Adibidez:
          \url{registry.gitlab.com/mu-bd-ce/devcontainers/python:latest}
        \item Hauei buruzko \textbf{dokumentazioa}:
          \url{https://mu-bd-ce.gitlab.io/devcontainers}
        \item Hauek nola erabili azaltzen duten \textbf{adibideak}.
      \end{itemize}
    \item Bigarren mailako web eta estatistika ikasgaietan \textbf{lehenengo saiakera} egin dugu.
  \end{itemize}
\end{frame}

\begin{frame}
  \frametitle{Gure Proposamena}
  \begin{itemize}
    \item Uztailean Devcontainer-ei buruzko \textbf{tailer} bat egitea.
    \item Hurrengo kurtsoan ekimen honetara \textbf{ikasgai gehiago batzea}.
    \item Git biltegia graduan erabiltzen ditugun \textbf{teknologiak estandarizatzeko} erabiltzea (adibidez, graduko Java bertsioa) \footnote{Biltegi hau lehenengo hurbilketa da; Devcontainer hauen alderdi guztiak eztabaidatzeko \textbf{irekita} daude.}.
  \end{itemize}
\end{frame}

\muback{Eskerrik asko}{Ibai Roman\\
  Software \& System Engineering\\
  Electronics and Computer Science department\\
  Mondragon Unibertsitatea - Faculty of Engineering\\
\href{www.mondragon.edu}{www.mondragon.edu}}

\end{document}